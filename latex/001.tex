\documentclass[12pt,a4paper]{report}  %紙張設定
\usepackage{xeCJK}%中文字體模組
\setCJKmainfont{標楷體} %設定中文字體
%\setCJKmainfont{MoeStandardKai.ttf}
\newfontfamily\sectionef{Times New Roman}%設定英文字體
%\newfontfamily\sectionef{Nimbus Roman}
\usepackage{enumerate}
\usepackage{titlesec}
\titleformat{\chapter}[display]
{\normalfont\fontsize{20}{22}\selectfont\bfseries\filcenter}
{\chaptertitlename\ \thechapter}{10pt}{\fontsize{18}{22}\selectfont}
\titlespacing*{\chapter}{0pt}{-10pt}{2pt}
\usepackage{amsmath,amssymb}%數學公式、符號
\usepackage{amsfonts} %數學簍空的英文字
\usepackage{graphicx, subfigure}%圖形
\usepackage{fontawesome5} %引用icon
\usepackage{type1cm} %調整字體絕對大小
\usepackage{textpos} %設定文字絕對位置
\usepackage[top=2.5truecm,bottom=2.5truecm,
left=3truecm,right=2.5truecm]{geometry}
\usepackage{titlesec} %目錄標題設定模組
\usepackage{titletoc} %目錄內容設定模組
\usepackage{textcomp} %表格設定模組
\usepackage{multirow} %合併行
%\usepackage{multicol} %合併欄
\usepackage{CJK} %中文模組
\usepackage{CJKnumb} %中文數字模組
\usepackage{wallpaper} %浮水印
\usepackage{listings} %引用程式碼
\usepackage{hyperref} %引用url連結
\usepackage{setspace}
\usepackage{lscape}%設定橫式
\lstset{language=Python, %設定語言
		basicstyle=\fontsize{10pt}{2pt}\selectfont, %設定程式內文字體大小
		frame=lines,	%設定程式框架為線
}
%\usepackage{subcaption}%副圖標
\graphicspath{{./../images/}} %圖片預設讀取路徑
\usepackage{indentfirst} %設定開頭縮排模組
\renewcommand{\figurename}{\Large 圖.} %更改圖片標題名稱
\renewcommand{\tablename}{\Large 表.}
\renewcommand{\lstlistingname}{\Large 程式.} %設定程式標示名稱
\hoffset=-5mm %調整左右邊界
\voffset=-8mm %調整上下邊界
\setlength{\parindent}{3em}%設定首行行距縮排
\usepackage{appendix} %附錄
\usepackage{diagbox}%引用表格
\usepackage{multirow}%表格置中

%--------------------封面-----------------------------%
\begin{document}

\begin{center}
\vspace*{1cm}
  \fontsize{18}{16}\selectfont \textbf{POLITECNICO DI TORINO}\par
\end{center}
\begin{center}
    \fontsize{18}{16}\selectfont \textbf{都靈理工大學}\\
\end{center}
% 在图像下方添加分界线
\noindent\rule{\textwidth}{0.4pt}
\vspace{0.5em}
\begin{center}
  \fontsize{12}{16}\selectfont \textbf{ANALYSIS OF THE ODOO SOFTWARE CAPABILITIES REGARDING 
PRODUCT LIFECYCLE MANAGEMENT, MANUFACTURING EXECUTION 
SYSTEMS AND THEIR INTEGRATION}\par
\end{center}
\begin{center}
   \fontsize{12}{22}\selectfont \textbf{ODOO 軟體功能分析產品生命週期管理、製造執行系統及其集成}
\end{center}
\vspace{1em}
\begin{figure}[h]
\vspace{2cm}
    \centering
    \includegraphics[width=0.33\textwidth]{logo} 
    \label{fig:logo}
\end{figure}
\vspace{2cm}
\noindent
\small\textbf{SUPERVISORS} \hfill \textbf{CANDIDATE}\\
\small\textbf{指導者} \hfill \textbf{申請人}\\
\small\textbf{Giulia Bruno } \hfill \textbf{Lucas Flabiano Perotti}\\
\small\textbf{朱莉婭·布魯諾} \hfill \textbf{盧卡斯·弗拉比亞諾·佩羅蒂}\\
\small\textbf{Franco Lombardi}\hspace{\fill} \\
\small\textbf{佛朗哥·隆巴迪}\hspace{\fill} \\
\noindent\rule{\textwidth}{0.4pt}
\begin{center}
\fontsize{12}{22}\selectfont {Academic Year 2020 – 2021}\\
\fontsize{12}{22}\selectfont {2020 – 2021 學年}
\end{center}
\thispagestyle{empty}
\newpage
%--------------------分頁-----------------------------%
\vspace*{\fill}
\vspace*{\fill}
\begin{raggedright}
    \fontsize{12}{14}\selectfont This work is subject to the Creative Commons Licence\\
    \fontsize{12}{14}\selectfont 本作品受知識共享授權約束\\[2ex]
    \fontsize{12}{12}\selectfont All Rights Reserved\\
    \fontsize{12}{12}\selectfont 版權所有\\

\end{raggedright}
\thispagestyle{empty}
\newpage
%--------------------致謝-----------------------------%
\pagenumbering{roman}
\setcounter{page}{2}

\begin{center}
 \fontsize{18}{16}\selectfont \textbf{ACKNOWLEDGMENTS}\\
\fontsize{18}{16}\selectfont \textbf{致謝}\par
\end{center}

\fontsize{14pt}{2.5pt}\sectionef
  {  I would like to thank Dr. Giulia Bruno for her expert advice and invitation to develop this project, as well as Emiliano Traini, for his extraordinary support in this thesis process.}。\\[1pt]

\fontsize{14pt}{5pt}\sectionef
  {  我要感謝朱莉婭·布魯諾博士的專家建議和邀請來開發這個
專案以及埃米利亞諾·特拉尼,感謝他在本論文過程中的非凡支持。}\\[15pt]

\fontsize{14pt}{2.5pt}\sectionef
  {  My most sincere gratitude to my parents, Julio and Michelle, who gave me everything, 
from my life to their extensive and unconditional support and encouragement; also, to my 
brothers and my fiancée Ana, who inspired me through all these years.}\\[1pt]

\fontsize{14pt}{5pt}\sectionef
  {  我最誠摯的感謝我的父母Julio和Michelle,他們給了我一切,從我的生活到他們廣泛、無條件的支持和鼓勵; 還有,對我的
兄弟們和我的未婚妻安娜這些年來一直激勵著我。}\\[15pt]

\fontsize{14pt}{2.5pt}\sectionef
{  My deepest thanks and appreciation to Icaro, Matt, and Maz, for their endless help and 
support throughout not just this project, but for all the other moments in which they pushed 
me to be better. Also, for those who have touched my life, being my greatest gifts, you all 
know who you are, and I am truly grateful for sharing special moments of my lif}\\[1pt]

\fontsize{14pt}{5pt}\sectionef
  {我對 Icaro、Matt 和 Maz 表示最深切的感謝和讚賞,感謝他們的無盡幫助,
不僅支持這個項目,還支持他們推動的所有時刻
我要變得更好。 另外,對於那些感動我生命的人,你們是我最好的禮物
知道你是誰,我真的很感激分享我生命中的特殊時刻}\\[15pt]
\newpage
%--------------------摘要-----------------------------%
\begin{center}
 \fontsize{18}{16}\selectfont \textbf{ABSTRACT}\\
\fontsize{18}{16}\selectfont \textbf{摘要}\par

\end{center}
\fontsize{16}{18}\sectionef \textbf
 {ANALYSIS  OF  THE ODOO  SOFTWARE  CAPABILITIES  REGARDING  
 PRODUCT  LIFECYCLE  MANAGEMENT,  MANUFACTURING  EXECUTION 
 SYSTEMS  AND  THEIR  INTEGRATION }。\\[2pt]
\fontsize{16}{18}\sectionef \textbf
  {(ODOO 軟體功能分析產品生命週期管理、製造執行系統及其集成)}。\\[15pt]

\fontsize{14pt}{2.5pt}\sectionef 
{ The second half of the 20th century had been marked for the advancements of computer 
technology in all aspects of production.}。\\[1pt]

\fontsize{14pt}{5pt}\sectionef
 {20世紀下半葉以電腦的進步為標誌生產各個環節的技術}\\[15pt]

\fontsize{14pt}{2.5pt}\sectionef 
{ The key feature of that statement is the undeniable truth that alongside the increased 
complexity allowed by computing power comes an ever increasing production of 
overwhelming amounts of information. }。\\[1pt]

\fontsize{14pt}{5pt}\sectionef
 {該聲明的關鍵特徵是不可否認的事實,即隨著增加的計算能力所允許的複雜性,帶來了不斷增加的產量與海量資訊。}\\[15pt]

\fontsize{14pt}{2.5pt}\sectionef 
{From separate perspectives of the industrial landscape, several systems were brewed by 
that sheer necessity for organization, automation and waste reduction focusing on that pool 
of useful data. }。\\[1pt]

\fontsize{14pt}{5pt}\sectionef
 {從工業景觀的不同角度來看,出於組織、自動化和減少浪費的絕對必要性,一些系統誕生了,這些系統專注於有用資料池。}\\[15pt]

\fontsize{14pt}{2.5pt}\sectionef 
{ERP (from a managerial perspective), MES (from a production perspective) and more 
recently PLM (from a strategic development/redevelopment perspective) emerged as 
information solutions tackling this problem from different angles. These solutions, however 
effective, are always plagued by the fundamental incompatibility between the tools that 
implement those systems.}。\\[1pt]

\fontsize{14pt}{5pt}\sectionef
 {ERP(從管理角度)、MES(從生產角度)以及最近的 PLM(從策略開發/再開發角度)作為
資訊解決方案從不同角度解決這個問題。 這些解決方案無論多麼有效,總是受到實現這些系統的工具之間根本不相容的困擾。}\\[15pt]

\fontsize{14pt}{2.5pt}\sectionef 
{This paper objectives revolve around analyzing the integration PLM and MES systems 
from a theoretical perspective and comment on the use of the Odoo software tool to 
implement said integration.}。\\[1pt]

\fontsize{14pt}{5pt}\sectionef
 {本文的目標是從理論角度分析 PLM 和 MES 系統的集成,並對使用 Odoo 軟體工具實現所述集成進行評論。}\\[15pt]
\newpage
\fontsize{14pt}{2.5pt}\sectionef 
{The Odoo software was described in detail (regarding its use for manufacturing 
envirorment) icluding how it implements PLM and MES. Then, the software was subjected 
to the simulation of a fictional firm devised in the molds of Industry 4.0. This company was
a fictional recently founded small case manufacturing company that uses plastic injection 
molding as their primary mean of production and uses additive manufacturing and fast 
prototyping as part of their business strategy.}。\\[1pt]

\fontsize{14pt}{5pt}\sectionef
 {詳細描述了 Odoo 軟體(關於其在製造環境中的使用),包括它如何實施 PLM 和 MES。 然後,該軟體對一家按照工業 4.0 模式設計的虛構公司進行模擬。 該公司是一家虛構的最近成立的小型箱體製造公司,使用塑膠注塑作為主要生產手段,並使用積層製造和快速原型製作作為其業務策略的一部分。}\\[15pt]

\fontsize{14pt}{2.5pt}\sectionef 
{Keywords: Product Life-Cycle Management, Product Life-Cycle Management, Odoo}。\\[1pt]

\fontsize{14pt}{5pt}\sectionef
 {關鍵字:產品生命週期管理、產品生命週期管理、Odoo}\\[15pt]
\newpage
%--------------------目錄-----------------------------%
%--------------------正文-----------------------------%
\input{01.tex}
\newpage
\input{02.tex}
\newpage
\input{03.tex}
\newpage
\input{04.tex}
\newpage
\end{document}